\documentclass[14pt,onecolumn]{extarticle}
\usepackage{CJKutf8}
\usepackage[overlap,CJK]{ruby}
\usepackage[table]{xcolor}
\renewcommand{\rubysep}{-0.2ex}
\renewcommand{\rubysize}{0.7}
\pagenumbering{gobble}
%\usepackage[landscape,left=3mm,right=3mm,top=3mm,bottom=3mm]{geometry}
\usepackage[portrait,left=15mm,right=3mm,top=3mm,bottom=3mm]{geometry}
\definecolor{verylightgray}{rgb}{0.9, 0.9, 0.9}
\begin{document}
\rowcolors{1}{white}{verylightgray}
\begin{CJK}{UTF8}{min}

\subsubsection{lesson 5 grammar notes (volume 2 page 31)}
\begin{tabular}{lrl}

gn1
&ongoing activity (formal)&verbている\\
&ongoing activity (relaxed)&verbてる\\
&ongoing activity (formal)&verbています\\
&ongoing activity (relaxed)&verbてます\\
&current state, due to previous occurance (formal)&verbている\\
&current state, due to previous occurance (relaxed)&verbてる\\
&current state, due to previous occurance (formal)&verbています\\
&current state, due to previous occurance (relaxed)&verbてます\\
\hline

gn2
&person/animal exists&person/animalがいる\\
&person/animal exists&person/animalがいます\\
&object exists&objectがある\\
&object exists&objectがあります\\
\hline

gn3
&indicates quantity of noun being verbed&nounをquantity+verb\\
&indicates quantity of noun being verbed&nounがquantity+verb\\
\hline

gn4
&this one&このnoun\\
&that one&そのnoun\\
&that one over there&あのnoun\\
&which one&どのnoun\\
\hline

gn5
&seems&i-adjective-stemぞうです\\
&seems&na-adjectiveぞうです\\
\hline

gn6
&describing a characteristic of something&topicわsubjectがadjectiveです\\
\hline

gn7
&in the (noun) times&nounのとき\\
&in the (i-adjective) times&i-adjectiveとき\\
&in the (na-adjective) times&na-adjectiveにとき\\
\hline

gn8
&to give&giverはrecipientにnounをあげる\\
&to give&giverがrecipientにnounをあげる\\
&to give (to me)&giverはrecipientにnounをくれる\\
&to give (to me)&giverがrecipientにnounをくれる\\
&to receive&recipientはgiverにnounをもらう\\
&to receive&recipientがgiverにnounをもらう\\
\hline

gn9
&when something occured&timeに\\
&when something occured&occasionに\\

\end{tabular}

\subsubsection{lesson 6 grammar notes (volume 2 page 61)}
\begin{tabular}{lrl}

gn1
&please&verbてください\\
&please (polite)&verbおねがいします\\
&please (relaxed)&verbて\\
\hline

gn2
&I think that&verb(plain)とおもう\\
&I think that&verb(plain)とおもいます\\
&I've been thinking that&verb(plain)とおもって(い)る\\
&I've been thinking that&verb(plain)とおもって(い)ます\\
\hline

gn3
&means by which something occurs&nounで\\
\hline

gn4
&to go somewhere having done something&verbていく\\
&to go somewhere having done something&verbていきます\\
&to come here having done something&verbてくる\\
&to come here having done something&verbてきます\\
\hline

gn5
&shall I&verbましょうか\\
\hline

gn6
&whatever is fine&なんでもいいです\\
&whenever is fine&いつでもいいです\\
&whereever is fine&どこでもいいです\\
\hline

gn7
&said that&indirect quote (plain)といった\\
&was saying that&indirect quote (plain)といって(い)た\\
&has been saying / keeps saying&indirect quote (plain)といって(い)る\\
\hline

gn8
&non-sequential action&verbて\\
\hline

gn9
&or&nounかnoun\\
\hline

gn10
&advanced preparation&verbておく\\
&advanced preparation&verbておいた\\
\hline

gn11
&turns out to be&nounになる\\
&turned out to be&nounになった\\
&turns out to be&na-adjectiveになる\\
&turned out to be&na-adjectiveになった\\
&turns out to be&i-adjective(stem)くなる\\
&turned out to be&i-adjective(stem)くなった\\
\hline

gn12
&noun2 from noun1&noun1からのnoun2\\
\hline

gn13
&noun1, noun2, etc (formal)&noun1やnoun2など\\
\hline

\end{tabular}

\subsubsection{lesson 7 grammar notes (volume 2 page 99)}

\begin{tabular}{lrl}

gn1&yet&まだverbて(い)ない\\
&yet&まだverbて(い)ません\\
&already&もうverbた\\
&already&もうverbました\\
\hline

gn2&adverbs of degree&
  とても・とっても\\
&&すごく・すっごく\\
&&けっこう\\
&&まあまあ\\
&&ちょう\\
&&あまり〜ない・あんまり〜ない\\
&&そんなに〜ない\\
&&ぜんぜん〜ない\\
&adverbs of frequency&
  いつも\\
&&よく\\
&&たまに\\
&&ときどき\\
&&たいてい\\
&&めったに〜ない\\
\hline

gn3&the 〜 one&〜の\\
\hline

gn4&to give multiple reasons&plainし、。。plainし、。。\\
\hline

gn5&familliar to both parties&あの\\
&not familliar to everyone&その\\
\hline

gn6&to try something&
  verbてみる\\
&(to see what it's like)&
  verbてみます\\
\hline

gn7&don't have to do&verbなくてもいい\\
\hline

gn8&yet/ever/already&plain/past form\\
&yet/ever/already&verbたことがある\\
&yet/ever/already&verbたことがあります\\
&not yet&verbたことはない\\
&not yet&verbたことはありません\\
\hline

gn9&best among 3 or more&
  nounがいちばんすき(だ)\\
&&nounがいちばんいい\\
\hline

gn10&&question word + も\\
&any \# of times, many times&なんでも\\
&any \# of people, many people&なんにんも\\
&any \# of things; many things&いくつも\\
\hline

gn11&both A and B&AもBも\\
&by A and by B&AでもBでも\\
&in A and in B&AにもBにも\\

\end{tabular}

\subsubsection{lesson 8 grammar notes (volume 2 page 132)}

\begin{tabular}{lrl}

gn1&to completion/finality(possibly with regret)&\\
&formal&verbてしまう\\
&relaxed&verbちゃう\\
\hline

gn2&asking for permission&\\
&Is it alright if I? May I? (relaxed)&verbて(も)いいですか\\
&formal&verbてもよろしいですか\\
&formal(less assertive)&verbてもよろしいでしょうか\\
&relaxed&verbて(も)いい?\\
\hline

gn3&negative request (please don't do ...)&\\
&formal (from authority)&verbないでください\\
&casual&verbないで\\
\hline

gn4&adverbing&\\
&&i-adjective stem + く\\
&&na-adjective (dictionary form) + に\\
\hline

gn5&before noun&nounの\ruby{前}{まえ}に\\
&after noun&nounの\ruby{後}{あと}で\\
&before doing...&verb(plain non-past)\ruby{前}{まえ}に\\
&after doing...&verb(plain past)\ruby{後}{あと}で\\
\hline

gn6&stating one's obligation&\\
&If I don't... things won't turn&\\
&out the way they should&\\
&must not (formal)&verbなければならない\\
&must not &verbなくてはならない\\
&must not (casual)&verbなくちゃならない\\
&must not&verbなきゃならない\\
&should not&verbなければいけない\\
&should not&verbなくてはいけない\\
&should not (relaxed)&verbなくちゃいけない\\
&should not (relaxed)&verbなきゃいけない\\
\hline

gn7&connecting clauses indicating reason&\\
&&verbて\\
&&i-adjectiveくて\\
&&na-adjectiveで\\
&&nounで\\
\hline

gn8&as many as&quantityも\\
&as long as&durationも\\
\hline

gn9&because of reason statement1,&statement1のでstatement2\\
&statement2 is the consequence(formal)&\\
\hline

\end{tabular}

\end{CJK}
\end{document}

