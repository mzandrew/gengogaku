\documentclass[extrafontsizes,14pt]{memoir}
\usepackage{CJKutf8}
\usepackage[overlap,CJK]{ruby}
\usepackage[table]{xcolor}
\renewcommand{\rubysep}{-0.2ex}
\renewcommand{\rubysize}{0.7}
\pagenumbering{gobble}
%\usepackage[landscape,left=3mm,right=3mm,top=3mm,bottom=3mm]{geometry}
\usepackage[portrait,left=15mm,right=3mm,top=3mm,bottom=3mm]{geometry}
\definecolor{verylightgray}{rgb}{0.9, 0.9, 0.9}
\begin{document}
\rowcolors{1}{white}{verylightgray}
\begin{CJK}{UTF8}{min}

\subsubsection{lesson 7 grammar notes (volume 2 page 99)}
\begin{tabular}{lrl}

gn1&yet/already&
  まだverbて(い)ない\\
&&まだverbて(い)ません\\
&&もうverbた\\
&&もうverbました\\
\hline

gn2&adverbs of degree&
  とても・とっても\\
&&すごく・すっごく\\
&&けっこう\\
&&まあまあ\\
&&ちょう\\
&&あまり〜ない・あんまり〜ない\\
&&そんなに〜ない\\
&&ぜんぜん〜ない\\
&adverbs of frequency&
  いつも\\
&&よく\\
&&たまに\\
&&ときどき\\
&&たいてい\\
&&めったに〜ない\\
\hline

gn3&the 〜 one&〜の\\
\hline

gn4&to give multiple reasons&plainし、。。plainし、。。\\
\hline

gn5&こ・そ・あ・ど&\\
&familliar to both parties&あの\\
&not familliar to everyone&その\\
\hline

gn6&to try something&
  verbてみる\\
&(to see what it's like)&
  verbてみます\\
\hline

gn7&don't have to do&verbなくてもいい\\
\hline

gn8&yet/ever/already (plain/past form)&\\
&yet/ever/already&verbたことがある\\
&yet/ever/already&verbたことがあります\\
&not yet&verbたことはない\\
&not yet&verbたことはありません\\
\hline

gn9&best among 3 or more&
  nounがいちばんすき(だ)\\
&&nounがいちばんいい\\
\hline

gn10&question word + も&\\
&any \# of times, many times&なんでも\\
&any \# of people, many people&なんにんも\\
&any \# of things; many things&いくつも\\
\hline

gn11&both A and B&AもBも\\
&by A and by B&AでもBでも\\
&in A and in B&AにもBにも\\

\end{tabular}

\end{CJK}
\end{document}

